%% =============================
%% GENERAL SETTINGS
%% =============================
\documentclass[12pt]{article}

\usepackage [margin=2.5cm]{geometry}
\usepackage {graphics}
\usepackage {xltxtra} 
\usepackage {xgreek} 
\usepackage {color}
\usepackage {hyperref}
\hypersetup {colorlinks}


\setmainfont[Mapping=tex-text]{Tahoma} 
\setlength{\parindent}{0cm} 				%% No paragraph indent

\definecolor{darkred}{rgb}{0.5,0,0}
\definecolor{darkgreen}{rgb}{0,0.5,0}
\definecolor{darkblue}{rgb}{0,0,0.5}

\hypersetup{ colorlinks,
linkcolor=darkblue,
filecolor=darkgreen,
urlcolor=darkblue,
citecolor=darkred }

%% =============================
%% DOCUMENT PROPERTIES
%% =============================
\title{{\Huge {\bf Easy!Appointments}} \\[0.3cm] Περιπτώσεις Χρήσης}
\author{Αλέξανδρος Τσελεγγίδης}
\date{Νοέμβριος 2012}

%% =============================
%% DOCUMENT CONTENT
%% =============================
\begin{document}
\maketitle 
\thispagestyle{empty} %% Απομάκρυνση page number από την πρώτη σελίδα
\pagebreak

%% ΠΕΡΙΠΤΩΣΕΙΣ ΧΡΗΣΗΣ ΔΙΑΧΕΙΡΙΣΤΗ
\section {Περιπτώσεις Χρήσης Διαχειριστή}
\subsection{Εγκατάσταση της εφαρμογής}
Αυτή η περίπτωση χρήσης περιλαμβάνει την ρύθμιση του server όπου θα τρέξει το Easy!Appointments και την δημιουργία του λογαριασμού του διαχειριστή.
\\[0.3cm]
Χαρακτήρες: Διαχειριστής, Σύστημα Βάσης Δεδομένων
\\[0.3cm]
Βαθμός σημαντικότητας: 9
\\[0.3cm]
Βαθμός δυσκολίας: 2

\subsection{Παραμετροποίηση της εφαρμογής}
Για να μπορέσει να λειτουργήσει η εφαρμογή σύμφωνα με την μορφή της επιχείρησης θα χρειαστεί να παραμετροποιηθεί από τον διαχειριστή. Η παραμετροποίηση περιλαμβάνει τα ωράρια λειτουργίας της επιχείρησης, την διαχείρηση των υπηρεσιών που θα ειναι διαθέσιμες προς το κοινό, καθώς και την διαχειρηση των πάροχων υπηρεσιών. Οι ρυθμίσεις αυτές θα μπορούν να μεταφέρονται από εγκατάσταση σε εγκατάσταση (portable).
\\[0.3cm]
Χαρακτήρες: Διαχειριστής, Σύστημα Βάσης Δεδομένων
\\[0.3cm]
Βαθμός σημαντικότητας: 8
\\[0.3cm]
Βαθμός δυσκολίας: 2

\subsection{Διαχείριση Ραντεβού}
Τα ραντεβού που θα κλείνουν οι πελάτες για τις διάφορες υπηρεσίες που προσφέρει η επιχείρηση θα εμφανίζονται πάνω σε ημερολόγια, τα οποία αντιπροσωπεύουν τα πλάνα των υπαλλήλων (πάροχων υπηρεσιών). Ο διαχειριστής θα μπορεί να κάνει οποιαδηποτε αλλαγή σε αυτά τα ραντεβού, αλλάζοντας έτσι το πλάνο ενός υπαλλήλου. Εκτός αυτού, ο διαχειριστής θα έχει την δυνατότητα να διαχειριστεί όλους τους πελάτες που έχουν καταχωρηθεί στο σύστημα.
\\[0.3cm]
Χαρακτήρες: Διαχειριστής, Σύστημα Βάσης Δεδομένων
\\[0.3cm]
Βαθμός σημαντικότητας: 10
\\[0.3cm]
Βαθμός δυσκολίας: 3

\subsection{Διαχείριση Πάροχων Υπηρεσίας}
Τις υπηρεσίες που προσφέρει η εταιρεία τις αναλαμβάνουν κάποιοι υπάλληλοι (ή ομάδες υπαλλήλων), οι οποίοι αναφέρονται στο σύστημα ως πάροχοι υπηρεσιών. Τα στοιχεία τους, τις αρμοδιότητές τους και τα δικαιώματα μέσα στο συστημα τα ορίζει μόνο ο διαχειριστής του συστήματος. Αποτελεί υπο-περίπτωση χρησης της παραμετροποίησης της εφαρμογής.
\\[0.3cm]
Χαρακτήρες: Διαχειριστής, Σύστημα Βάσης Δεδομένων
\\[0.3cm]
Βαθμός σημαντικότητας: 7
\\[0.3cm]
Βαθμός δυσκολίας: 1

\subsection{Διαχείριση Υπηρεσιών}
Οι πελάτες που θα επισκεπτονται τον ιστότοπο του Easy!Appointments της επιχείρησης θα κλείνουν ραντεβού για συγκεκριμένες υπηρεσίες. Το ποιές υπηρεσίες θα είναι διαθέσιμές και ποιοί πάροχοι υπηρεσιών μπορούν να εξυπηρετησουν τι, το διαχειρίζεται ο διαχειριστής του συστήματος. Αποτελεί υπο-περίπτωση χρησης της παραμετροποίησης της εφαρμογής.
\\[0.3cm]
Χαρακτήρες: Διαχειριστής, Σύστημα Βάσης Δεδομένων
\\[0.3cm]
Βαθμός σημαντικότητας: 6 
\\[0.3cm]
Βαθμός δυσκολίας: 1

\subsection{Διαχείριση Γραμματειών}
Μερικές επιχειρήσεις με μεγάλο αριθμό υπαλλήλων, μπορεί να αναθέσουν το καθήκον της διαχείρισης ραντεβού και πελατών σε έναν ή περισσότερους γραμματείς. Οι γραμματείς που θα μπορούν να μπαίνουν στο σύστημα για να διαχειρίζονται τα ραντεβού των πάροχων θα καταχωρούνται από τον διαχειριστή του συστήματος.
\\[0.3cm]
Χαρακτήρες: Διαχειριστής, Σύστημα Βάσης Δεδομένων
\\[0.3cm]
Βαθμός σημαντικότητας: 6 
\\[0.3cm]
Βαθμός δυσκολίας: 1

%% ΠΕΡΙΠΤΩΣΕΙΣ ΧΡΗΣΗΣ ΠΑΡΟΧΟΥ ΥΠΗΡΕΣΙΩΝ 
\section {Περιπτώσεις Χρήσης Πάροχου Υπηρεσιών}
\subsection{Διαχείριση Ραντεβού}
Ο πάροχος υπηρεσίας, αφού του έχουν δοθεί τα κατάλληλα δικαιώματα από τον διαχειριστή θα μπορεί να διαχειρίζεται τα ραντεβού που αντιστοιχούν σε αυτόν. Για παράδειγμα θα μπορεί να αλλάξει την ημερομηνία και την ώρα, την διάρκεια και τα υπόλοιπα χαρακτηριστικά ενός ραντεβού.
\\[0.3cm]
Χαρακτήρες: Πάροχος Υπηρεσίων, Σύστημα Βάσης Δεδομένων
\\[0.3cm]
Βαθμός σημαντικότητας: 7
\\[0.3cm]
Βαθμός δυσκολίας: 1

\subsection{Διαχείριση Ημερολογιακού Πλάνου}
Σε κάθε πάροχο υπηρεσιών αντιστοιχεί ένα ημερολογιακό πλάνο, το οποίο περιέχει τα ραντεβού που έχουν κανονιστεί μεταξύ αυτού και των πελατών. Αυτό το πλάνο μπορεί ο χρήστης να το δει, να το επεξεργαστεί και να το συγχρονίσει με άλλες υπηρεσίες ημερολογίων (Google Calendar). Εκτός αυτού μπορεί να ορίσει ένα πλάνο ως πρότυπο και έτσι αυτό να αποτελεί την βάση για κάθε νέα εβδομάδα.
\\[0.3cm] 
Χαρακτήρες: Πάροχος Υπηρεσιών, Σύστημα Βάσης Δεδομένων, Google Calendar
\\[0.3cm]
Βαθμός σημαντικότητας: 10
\\[0.3cm]
Βαθμός δυσκολίας: 3

\subsection{Λήψη Ειδοποιήσεων από το Σύστημα}
Όταν γίνονται νέες κρατήσεις ραντεβού, αλλά και όταν πραγματοποιούνται αλλαγές στο πλάνο ενός παρόχου υπηρεσίας πρέπει όλοι οι εμπλεκόμενοι να ενημερωθούν με κάποιον τρόπο. Για αυτόν τον λόγο το Easy!Appointments θα περιέχει ένα υποσύστημα το οποίο θα αποστέλνει Email σε όποιον χρήστη ενδιαφέρει άμεσα η αλλαγή.
\\[0.3cm] 
Χαρακτήρες: Σύστημα Ειδοποιήσεων, Υπηρεσία SMS
\\[0.3cm]
Βαθμός σημαντικότητας: 9
\\[0.3cm]
Βαθμός δυσκολίας: 2

%% ΠΕΡΙΠΤΩΣΕΙΣ ΧΡΗΣΗΣ ΠΕΛΑΤΗ
\section {Περιπτώσεις Χρήσης Πελάτη}
\subsection{Κράτηση Ραντεβού}
Αφού ο πελάτης βρει την ημερομηνία και την ώρα που τον βολεύει για το ραντεβού του, θα έχει την δυνατότητα να πραγματοποιήσει μια κράτηση, καταχωρόντας τα στοιχεία του στο σύστημα. Με το πέρας αυτής της διαδικασίας το επιλεγμένο χρονικό διάστημα έχει δεσμευτεί και δεν μπορεί κάποιος άλλος πελάτης να το πάρει. 
\\[0.3cm] 
Χαρακτήρες: Πελάτης, Σύστημα Βάσης Δεδομένων
\\[0.3cm]
Βαθμός σημαντικότητας: 9
\\[0.3cm]
Βαθμός δυσκολίας: 2

\subsection{Επεξεργασία - Ακύρωση Ραντεβού}
Είναι πολύ σημαντικό να υπάρχει η δυνατοτητα ο ίδιος πελάτης να μπορεί να ακυρώσει το ραντεβού του (σημειώνοντας τον λόγο αν θέλει) ή κάποιος άλλος χρήστης με τα ανάλογα δικαιώματα να μπορεί να αλλάξει ημερομηνία και να κάνει τροποποιήσεις στο ραντεβού. Μόλις ο πελάτης ολοκληρώσει την κράτηση ενός ραντεβού, θα του έρχεται ένα email το οποίο θα περιέχει έναν σύνδεσμο (μοναδικό) ο οποίος θα οδηγεί στην σελίδα επεξεργασίας του ραντεβού. Σε αυτήν την σελίδα ο πελάτης θα μπορεί να πραγματοποιήσει αλλαγές πάνω σε ένα υπάρχον ραντεβού (αλλαγή ημερομηνίας, υπηρεσίας ή και ακύρωση ραντεβού).
\\[0.3cm] 
Χαρακτήρες: Πελάτης, Σύστημα Βάσης Δεδομένων, Συστημά Ειδοποιήσεων
\\[0.3cm]
Βαθμός σημαντικότητας: 9
\\[0.3cm]
Βαθμός δυσκολίας: 3

\subsection{Ειδοποιήσεις Συστήματος}
Παρόμοια με την προαναφερθέντα περίπτωση χρήσης, ο πελάτης θα ενημερώνεται για οποιαδήποτε αλλαγή έχει γίνει σε κάποιο ραντεβού του. 
\\[0.3cm] 
Χαρακτήρες: Πάροχος Υπηρεσιών, Σύστημα Βάσης Δεδομένων, Google Calendar
\\[0.3cm]
Βαθμός σημαντικότητας: 8
\\[0.3cm]
Βαθμός δυσκολίας: 2

% ΠΕΡΙΠΤΩΣΕΙΣ ΧΡΗΣΗΣ ΓΡΑΜΜΑΤΕΑ
\section {Περιπτώσεις Χρήσης Γραμματέα}
\subsection{Διαχείριση Ημερολογιακών Πλάνων}
Ο χρήστης γραμματέας μπορεί να πραγματοποιήσει αλλαγές στα ραντεβού ενός ή περισσότερων πάροχων υπηρεσίας. 
\\[0.3cm] 
Χαρακτήρες: Γραμματέας, Σύστημα Βάσης Δεδομένων
\\[0.3cm]
Βαθμός σημαντικότητας: 7
\\[0.3cm]
Βαθμός δυσκολίας: 2

\subsection{Διαχείριση Πελατών}
Ο χρηστης γραμματέας μπορεί να διαχειρίζεται τους καταχωρημένους πελάτες, διευκολύνοντας έτσι την λειτουργία της επιχείρησης και την εξυπηρέτηση των πελατών.
\\[0.3cm] 
Χαρακτήρες: Γραμματέας, Σύστημα Βάσης Δεδομένων
\\[0.3cm]
Βαθμός σημαντικότητας: 6
\\[0.3cm]
Βαθμός δυσκολίας: 1

\end{document}