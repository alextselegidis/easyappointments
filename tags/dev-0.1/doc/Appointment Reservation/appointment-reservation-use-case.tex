%% =============================
%% GENERAL SETTINGS
%% =============================
\documentclass[12pt]{article}

\usepackage [margin=2.5cm]{geometry}
\usepackage {graphics}
\usepackage {xltxtra} 
\usepackage {xgreek} 
\usepackage {color}
\usepackage {hyperref}
\hypersetup {colorlinks}


\setmainfont[Mapping=tex-text]{Tahoma} 
\setlength{\parindent}{0cm} 				%% No paragraph indent

\definecolor{darkred}{rgb}{0.5,0,0}
\definecolor{darkgreen}{rgb}{0,0.5,0}
\definecolor{darkblue}{rgb}{0,0,0.5}

\hypersetup{ colorlinks,
linkcolor=darkblue,
filecolor=darkgreen,
urlcolor=darkblue,
citecolor=darkred }

%% =============================
%% DOCUMENT PROPERTIES
%% =============================
\title{{\Huge {\bf Easy!Appointments}} \\[0.3cm] Περίπτωση Χρήσης Κράτηση Ραντεβού}
\author{Αλέξανδρος Τσελεγγίδης}
\date{Απρίλιος 2013}

%% =============================
%% DOCUMENT CONTENT
%% =============================
\begin{document}
\maketitle 
\thispagestyle{empty} %% Απομάκρυνση page number από την πρώτη σελίδα

%% ΑΝΑΛΥΤΙΚΗ ΠΕΡΙΓΡΑΦΗ ΚΡΑΤΗΣΗΣ ΡΑΝΤΕΒΟΥ
{\bf Βασική Ροή}
\\[0.3cm]
Ο χρήστης μπαίνει στην σελίδα κράτησης ραντεβού και επιλέγει την υπηρεσία και τον πάροχο που τον ενδιαφέρει. Στην συνέχεια θα χρειαστεί να επιλέξει μια από τις διαθέσιμες ημερομηνίες και ώρες για να κλείσει το ραντεβού του. Αφού γίνει και αυτό θα χρειαστεί να συμπληρώσει τα στοιχεία του έτσι ώστε να μπορέσει η εταιρεία να έρθει σε επαφή μαζί του αν χρειαστεί. Τέλος ένα email θα σταλθεί πίσω στον πελάτη ότι το ραντεβού του έχει καταχωρηθεί με επιτυχία.
\\[0.3cm]

{\bf Εναλλακτικές Ροές}
\begin{itemize}
\item Αν ο πελάτης αργήσει να επιλέξει ημερομηνία και στο ενδιάμεσο τον προλάβει ένας άλλος, θα πρέπει να επιστραφεί μήνυμα το οποίο θα τον προτρέψει να βρει άλλη ημερομηνία και ώρα για το ραντεβού του.
\item Όταν ο πελάτης συμπληρώνει τα στοιχεία του και αφήσει κενό ένα πεδίο το οποίο είναι υποχρεωτικό για να ολοκληρωθεί η διαδικασία, θα εμφανιστεί μήνυμα το οποίο θα τον προτρέψει να συμπληρώσει όλα τα υποχρεωτικά πεδία.
\end{itemize}


\end{document}