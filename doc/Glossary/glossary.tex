%% =============================
%% GENERAL SETTINGS
%% =============================
\documentclass[12pt]{article}

\usepackage [margin=2.5cm]{geometry}
\usepackage {graphics}
\usepackage {xltxtra} 
\usepackage {xgreek} 
\usepackage {color}
\usepackage {hyperref}
\hypersetup {colorlinks}


\setmainfont[Mapping=tex-text]{Tahoma} 
\setlength{\parindent}{0cm} 				%% No paragraph indent

\definecolor{darkred}{rgb}{0.5,0,0}
\definecolor{darkgreen}{rgb}{0,0.5,0}
\definecolor{darkblue}{rgb}{0,0,0.5}

\hypersetup{ colorlinks,
linkcolor=darkblue,
filecolor=darkgreen,
urlcolor=darkblue,
citecolor=darkred }

%% =============================
%% DOCUMENT PROPERTIES
%% =============================
\title{{\Huge {\bf Easy!Appointments}} \\[0.3cm] Γλωσσάρι Όρων}
\author{Αλέξανδρος Τσελεγγίδης}
\date{Νοέμβριος 2012}

%% =============================
%% DOCUMENT CONTENT
%% =============================
\begin{document}
\maketitle 
\thispagestyle{empty} %% Απομάκρυνση page number από την πρώτη σελίδα
\pagebreak

%% ΓΛΩΣΣΑΡΙ
\section {Γλωσσάρι Όρων} %% Ο αστερίσκος δεν βάζει νούμερο στο section
\subsection {Διαχειριστής}
Ο διαχειριστής του συστήματος είναι ο χρήστης ο οποίος έχει όλα τα δικαιώματα αλλαγών και ρυθμίσεων του Easy!Appointments. Μπορεί να ορίσει νέες υπηρεσίες και πάροχους υπηρεσίας, να ρυθμίσει το σύστημα ειδοποιήσεων και να εκτελέση όλες τις δυνατές διαδικασίες διαχείρησης των δεδομένων. 

\subsection {Πάροχος Υπηρεσίας}
Ο πάροχος υπηρεσίας είναι η οντότητα που εξυπηρετεί μια ή περισσότερες υπηρεσίες. Μπορεί να αντιπροσωπεύει ένα άτομο ή μια ομάδα ατόμων. Σε κάθε περίπτωση όμως διαχειρίζεται από έναν χρήστη.

\subsection {Πελάτης}
Ο πελάτης αφού δει τις διαθέσιμες ημερομηνίες και ώρες για τις επιλεγμένες υπηρεσίες και παρόχους, μπορεί να κλείνει ραντεβού με την επιχείρηση. Αν γίνει οποιαδήποτε αλλαγή σε κάποιο ραντεβού του πελάτη τότε αυτός μπορεί να ενημερωθεί είτε με email, είτε με sms (εφόσον έχει ρυθμιστεί η υπηρεσία).

\subsection {Ημερολογιακό Πλάνο Πάροχου Υπηρεσιών}
Από την στιγμή που κλείνονται ραντεβού σε έναν πάροχο υπηρεσιών το ημερολογιακό του πλάνο αρχίζει να γεμίζει από χρονικά διαστήματα, τα οποία είναι δεσμευμένα και αντιπροσωπεύουν συναντήσεις με τους πελάτες. Εκτός αυτού υπάρχει και η δυνατότητα να τεθεί ένα ανενεργό χρονικό διάστημα, στο οποίο ο συγκεκριμένος πάροχος δεν θα είναι διαθέσιμος έτσι ώστε να μην μπορούν οι πελάτες να κλείνουν ραντεβού σε αυτό το διάστημα. Αυτό το πλάνο μπορεί να συγχρονιστεί με το Google Calendar έτσι ώστε να είναι προσβάσιμο και από άλλες υπηρεσίες.

\end{document}